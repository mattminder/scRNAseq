\documentclass[10pt,conference,compsocconf]{IEEEtran}

\usepackage{hyperref}
\usepackage{graphicx}	% For figure environment
\usepackage{amsmath}
\usepackage{amssymb}
\usepackage{float}

\newcommand{\beginsupplement}{%
	\setcounter{table}{0}
	\renewcommand{\thetable}{S\arabic{table}}%
	\setcounter{figure}{0}
	\renewcommand{\thefigure}{S\arabic{figure}}%
}

\begin{document}
\title{CS-433 Machine Learning Project 1}

\author{
  Matthias Minder, Zora Oswald, Silvan Stettler\\
}

\maketitle

\begin{abstract}
Abstract...
\end{abstract}

\section*{Introduction} 
The recently developed method of single cell RNA sequencing (scRNA-seq) allows to measure the amount of RNA from a specific gene on a single cell resolution. This gives valuable insights into the properties and cellular function of single cells in a whole population of cells. High throughput methods allow to analyze thousands of cells simultaneously. scRNA-seq results in so-called read-count matrices (RCMs) which indicate for a given gene how many times it was expressed in a given cell.
\par 
The emergence of scRNA-seq has lead to the discovery of many new cell types based on their gene expression profile. The question thus naturally arises whether it would be possible to predict cell types using a machine learning approach. Of special interest is the detection and \textit{de novo} discovery of stem cells in tissues for which no stem cell population has been characterized. This project aims to create a classifier able to assign a "stemness" score to cells based on scRNA-seq data. 
\par
However, several challenges are associated with working with scRNA-seq data. The three most predominant issues are the following: Differences in procedures give rise to batch effects, leading to substantial overall differences between datasets of different sources. Secondly, cell-type annotation is based on clusters of the dataset itself, which may lead to biases when basing analysis on the annotated clusters. Thirdly, the obtained data is perturbed by drop-outs, where the read-counts for a given gene is measured to be zero, even if there were a signal. The result is that read count matrices are very sparse. 
\par
Several additional issues arises from a machine learning perspective: Since the genes interact with each other, they are highly correlated. However, due to the noisy nature of the data, identifying highly correlated features is hard. The scRNA-seq data can thus be thought of to be on a sub-dimensional manifold. Accurate representation of this manifold is key for obtaining results that generalize well. Moreover, a good such representation also reduces the impact of drop-outs, since correlated genes should correct the drop-out of one-another. This nature of scRNA-data suggests that application of dimensionality reduction prior to method training will yield results that perform better on independent test data. 
\par

\section*{Methods}
\subsection{Set Creation}


\section*{Results}

\section*{Conclusion}

%%% Bibliography
%\bibliographystyle{IEEEtran}
%\bibliography{literature-project2}


\end{document}
